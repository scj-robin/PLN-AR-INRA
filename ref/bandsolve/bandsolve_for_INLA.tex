% Options for packages loaded elsewhere
\PassOptionsToPackage{unicode}{hyperref}
\PassOptionsToPackage{hyphens}{url}
%
\documentclass[
]{article}
\usepackage{amsmath,amssymb}
\usepackage{iftex}
\ifPDFTeX
  \usepackage[T1]{fontenc}
  \usepackage[utf8]{inputenc}
  \usepackage{textcomp} % provide euro and other symbols
\else % if luatex or xetex
  \usepackage{unicode-math} % this also loads fontspec
  \defaultfontfeatures{Scale=MatchLowercase}
  \defaultfontfeatures[\rmfamily]{Ligatures=TeX,Scale=1}
\fi
\usepackage{lmodern}
\ifPDFTeX\else
  % xetex/luatex font selection
\fi
% Use upquote if available, for straight quotes in verbatim environments
\IfFileExists{upquote.sty}{\usepackage{upquote}}{}
\IfFileExists{microtype.sty}{% use microtype if available
  \usepackage[]{microtype}
  \UseMicrotypeSet[protrusion]{basicmath} % disable protrusion for tt fonts
}{}
\makeatletter
\@ifundefined{KOMAClassName}{% if non-KOMA class
  \IfFileExists{parskip.sty}{%
    \usepackage{parskip}
  }{% else
    \setlength{\parindent}{0pt}
    \setlength{\parskip}{6pt plus 2pt minus 1pt}}
}{% if KOMA class
  \KOMAoptions{parskip=half}}
\makeatother
\usepackage{xcolor}
\usepackage[margin=1in]{geometry}
\usepackage{color}
\usepackage{fancyvrb}
\newcommand{\VerbBar}{|}
\newcommand{\VERB}{\Verb[commandchars=\\\{\}]}
\DefineVerbatimEnvironment{Highlighting}{Verbatim}{commandchars=\\\{\}}
% Add ',fontsize=\small' for more characters per line
\usepackage{framed}
\definecolor{shadecolor}{RGB}{248,248,248}
\newenvironment{Shaded}{\begin{snugshade}}{\end{snugshade}}
\newcommand{\AlertTok}[1]{\textcolor[rgb]{0.94,0.16,0.16}{#1}}
\newcommand{\AnnotationTok}[1]{\textcolor[rgb]{0.56,0.35,0.01}{\textbf{\textit{#1}}}}
\newcommand{\AttributeTok}[1]{\textcolor[rgb]{0.13,0.29,0.53}{#1}}
\newcommand{\BaseNTok}[1]{\textcolor[rgb]{0.00,0.00,0.81}{#1}}
\newcommand{\BuiltInTok}[1]{#1}
\newcommand{\CharTok}[1]{\textcolor[rgb]{0.31,0.60,0.02}{#1}}
\newcommand{\CommentTok}[1]{\textcolor[rgb]{0.56,0.35,0.01}{\textit{#1}}}
\newcommand{\CommentVarTok}[1]{\textcolor[rgb]{0.56,0.35,0.01}{\textbf{\textit{#1}}}}
\newcommand{\ConstantTok}[1]{\textcolor[rgb]{0.56,0.35,0.01}{#1}}
\newcommand{\ControlFlowTok}[1]{\textcolor[rgb]{0.13,0.29,0.53}{\textbf{#1}}}
\newcommand{\DataTypeTok}[1]{\textcolor[rgb]{0.13,0.29,0.53}{#1}}
\newcommand{\DecValTok}[1]{\textcolor[rgb]{0.00,0.00,0.81}{#1}}
\newcommand{\DocumentationTok}[1]{\textcolor[rgb]{0.56,0.35,0.01}{\textbf{\textit{#1}}}}
\newcommand{\ErrorTok}[1]{\textcolor[rgb]{0.64,0.00,0.00}{\textbf{#1}}}
\newcommand{\ExtensionTok}[1]{#1}
\newcommand{\FloatTok}[1]{\textcolor[rgb]{0.00,0.00,0.81}{#1}}
\newcommand{\FunctionTok}[1]{\textcolor[rgb]{0.13,0.29,0.53}{\textbf{#1}}}
\newcommand{\ImportTok}[1]{#1}
\newcommand{\InformationTok}[1]{\textcolor[rgb]{0.56,0.35,0.01}{\textbf{\textit{#1}}}}
\newcommand{\KeywordTok}[1]{\textcolor[rgb]{0.13,0.29,0.53}{\textbf{#1}}}
\newcommand{\NormalTok}[1]{#1}
\newcommand{\OperatorTok}[1]{\textcolor[rgb]{0.81,0.36,0.00}{\textbf{#1}}}
\newcommand{\OtherTok}[1]{\textcolor[rgb]{0.56,0.35,0.01}{#1}}
\newcommand{\PreprocessorTok}[1]{\textcolor[rgb]{0.56,0.35,0.01}{\textit{#1}}}
\newcommand{\RegionMarkerTok}[1]{#1}
\newcommand{\SpecialCharTok}[1]{\textcolor[rgb]{0.81,0.36,0.00}{\textbf{#1}}}
\newcommand{\SpecialStringTok}[1]{\textcolor[rgb]{0.31,0.60,0.02}{#1}}
\newcommand{\StringTok}[1]{\textcolor[rgb]{0.31,0.60,0.02}{#1}}
\newcommand{\VariableTok}[1]{\textcolor[rgb]{0.00,0.00,0.00}{#1}}
\newcommand{\VerbatimStringTok}[1]{\textcolor[rgb]{0.31,0.60,0.02}{#1}}
\newcommand{\WarningTok}[1]{\textcolor[rgb]{0.56,0.35,0.01}{\textbf{\textit{#1}}}}
\usepackage{graphicx}
\makeatletter
\def\maxwidth{\ifdim\Gin@nat@width>\linewidth\linewidth\else\Gin@nat@width\fi}
\def\maxheight{\ifdim\Gin@nat@height>\textheight\textheight\else\Gin@nat@height\fi}
\makeatother
% Scale images if necessary, so that they will not overflow the page
% margins by default, and it is still possible to overwrite the defaults
% using explicit options in \includegraphics[width, height, ...]{}
\setkeys{Gin}{width=\maxwidth,height=\maxheight,keepaspectratio}
% Set default figure placement to htbp
\makeatletter
\def\fps@figure{htbp}
\makeatother
\setlength{\emergencystretch}{3em} % prevent overfull lines
\providecommand{\tightlist}{%
  \setlength{\itemsep}{0pt}\setlength{\parskip}{0pt}}
\setcounter{secnumdepth}{5}
\ifLuaTeX
  \usepackage{selnolig}  % disable illegal ligatures
\fi
\usepackage{bookmark}
\IfFileExists{xurl.sty}{\usepackage{xurl}}{} % add URL line breaks if available
\urlstyle{same}
\hypersetup{
  pdftitle={Bandsolve for INLA},
  pdfauthor={G. Nuel},
  hidelinks,
  pdfcreator={LaTeX via pandoc}}

\title{Bandsolve for INLA}
\author{G. Nuel}
\date{28 juin, 2024}

\begin{document}
\maketitle

The package \(\texttt{bandsolve}\) to solve efficiently \(Ax=b\) where
\(A\) is a symmetric band matrix. Complexity in \(\mathcal{O}(n L)\)
where \(L\) is the number of bands. Also usable to obtain the inverse of
\(A\) (by solving with \(b\) base vectors).

\begin{Shaded}
\begin{Highlighting}[]
\CommentTok{\#install.packages("remotes")}
\CommentTok{\#remotes::install\_github("Monneret/bandsolve")}
\FunctionTok{require}\NormalTok{(bandsolve)}
\NormalTok{n}\OtherTok{=}\DecValTok{20}
\NormalTok{A}\OtherTok{=}\FunctionTok{matrix}\NormalTok{(}\DecValTok{0}\NormalTok{,n,}\DecValTok{2}\NormalTok{)}
\NormalTok{A[,}\DecValTok{1}\NormalTok{]}\OtherTok{=}\FloatTok{3.0}\NormalTok{; A[}\SpecialCharTok{{-}}\NormalTok{n,}\DecValTok{2}\NormalTok{]}\OtherTok{=}\SpecialCharTok{{-}}\FloatTok{1.0}
\NormalTok{b}\OtherTok{=}\FunctionTok{rep}\NormalTok{(}\FloatTok{1.0}\NormalTok{,n)}
\NormalTok{x}\OtherTok{=}\FunctionTok{bandsolve}\NormalTok{(A,b)}
\FunctionTok{norm}\NormalTok{(}\FunctionTok{rot2mat}\NormalTok{(A)}\SpecialCharTok{\%*\%}\NormalTok{x}\SpecialCharTok{{-}}\NormalTok{b)}
\end{Highlighting}
\end{Shaded}

\begin{verbatim}
## [1] 3.774758e-15
\end{verbatim}

\begin{Shaded}
\begin{Highlighting}[]
\CommentTok{\#bandsolve\_cpp(A,as.matrix(b))}
\end{Highlighting}
\end{Shaded}

When there is only one band off diagonal, the determinant of \(A\) can
also be computed efficiently using the recursion: \[
D_i=A[i,1]\times  D_{i+1}-A[i,2]^2 \times D_{i+2}
\] where \(D_i=\det(A[i:n,i:n])\). Log-scale implementation possible.

The log of the determinant is also easy to obtain:

\begin{Shaded}
\begin{Highlighting}[]
\CommentTok{\# log determinant}
\NormalTok{A}\OtherTok{=}\FunctionTok{matrix}\NormalTok{(}\DecValTok{0}\NormalTok{,n,}\DecValTok{3}\NormalTok{)}
\NormalTok{A[,}\DecValTok{1}\NormalTok{]}\OtherTok{=}\FloatTok{3.0}\NormalTok{; A[}\SpecialCharTok{{-}}\NormalTok{n,}\DecValTok{2}\NormalTok{]}\OtherTok{=}\SpecialCharTok{{-}}\FloatTok{1.0}\NormalTok{; A[}\DecValTok{1}\SpecialCharTok{:}\NormalTok{(n}\DecValTok{{-}2}\NormalTok{),}\DecValTok{3}\NormalTok{]}\OtherTok{=}\SpecialCharTok{+}\FloatTok{1.0}

\NormalTok{bandlogdet}\OtherTok{=}\ControlFlowTok{function}\NormalTok{(A) \{}
  \CommentTok{\# copy matrix to avoid in{-}place computation}
\NormalTok{  Amem}\OtherTok{=}\FunctionTok{matrix}\NormalTok{(}\ConstantTok{NA}\NormalTok{,}\FunctionTok{nrow}\NormalTok{(A),}\FunctionTok{ncol}\NormalTok{(A))}
\NormalTok{  Amem[]}\OtherTok{=}\NormalTok{A[]}
  \CommentTok{\# call LDL}
  \FunctionTok{invisible}\NormalTok{(}\FunctionTok{LDL}\NormalTok{(Amem))}
  \CommentTok{\# return res}
  \FunctionTok{return}\NormalTok{(}\FunctionTok{sum}\NormalTok{(}\FunctionTok{log}\NormalTok{(Amem[,}\DecValTok{1}\NormalTok{])))}
\NormalTok{\}}

\FunctionTok{c}\NormalTok{(}\FunctionTok{log}\NormalTok{(}\FunctionTok{det}\NormalTok{(}\FunctionTok{rot2mat}\NormalTok{(A))),}\FunctionTok{bandlogdet}\NormalTok{(A))}
\end{Highlighting}
\end{Shaded}

\begin{verbatim}
## [1] 17.68803 17.68803
\end{verbatim}

\end{document}
